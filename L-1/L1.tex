%%%%%%%%%%%%%%%%%%%%%%%%%%%%%%%%%

\chapter{测量单位与记数法}
%%%%%%%%%%%%%%%%%%%%%%%%%%%%%%%%%

\section{国际单位制 (SI)}
%%%%%%%%%%%%%%%%%%%%%%%%%%%%%%%%%
\nameref{tab:SI} 包括7个基本单位。

电气电子科学的基本单位是安培,它是电流的单位。用字母 \emph{I} 表示电流(强度),用
符号 \emph{A} 表示安培。安培在其定义中使用了时间的基本单位(秒),像这样使用基
本单位组合而成的单位,称为\textbf{导出单位}\sidenote{除了两个辅助单位, 球面度 (sr)、弧度 (rad),
现已并入导出单位。}。反之,则为\textbf{基本单位}。

一般来说,斜体字母代表物理量,非斜体(罗马)字母代表物理量的单位。

\begin{margintable}[2cm]
\begin{tabular}{lc}
\toprule
\multicolumn{2}{c}{\normalsize 基本物理量}      \\
\midrule
名称    & 符号 \\
时间    & $t$   \\                
长度    & $l$, $x$, $r$, etc. \\
质量    & $m$          \\
电流    & $I$, $i$       \\       
热力学温度 & $T$           \\        
物质的量  & $n$         \\
发光强度  & $I_v$   \\       
\bottomrule
\\
\\
\\
\\
\toprule
\multicolumn{2}{c}{\normalsize 基本单位} \\
\midrule
名称         & 符号         \\
秒           & s          \\
米           & m          \\
千克          & kg         \\
安培          & A          \\
开尔文         & K          \\
摩尔          & mol        \\
坎德拉         & cd        \\
\bottomrule
\end{tabular}
\caption[SI]{\tiny 国际单位制 (SI) 的基本物理量和基本单位}
\label{tab:SI}
\end{margintable}

\subsection{常用导出单位}
123

456

789
%%%%%%%%%%%%%%%%%%%%%%%%%%%%%%%%%

\section{工程记数法和公制词头}
%%%%%%%%%%%%%%%%%%%%%%%%%%%%%%%%%
工程记数法是科学记数法的一种特殊形式,在技术领域中广泛应用于表示较大和较小
的量值。工程记数法要求指数必须是3的整数倍,这样就可以方便地与公制词头相结合。
公制词头是国际单位制中用于表示数量级的前缀,它们表示10的整数次幂,如千 (10$^3$) 、兆 (10$^6$) 等。
%% 
\subsection{工程记数法}
在工程记数法中,一个数的小数点左边可以有1 $\sim$ 3位数字,指数部分须是3的倍数。

例如,在工程记数法中,$33\;000$表示为$33\times 10^3$,而在科学记数法中,它表示为$3.3\times 10^4$。
另一个例子是,在工程记数法中,$0.0\;45$表示为$45\times 10^-3$, 而在科学记数法中,它表示为
 $4.5\times 10^-2$。
 
%% 
\subsection{公制词头}

\begin{table*}
\begin{tabular}{cccccc}
\toprule
{\normalsize 公制词头名称}&{\normalsize 符号}&{\normalsize 指数幂}&{\normalsize 公制词头名
称}&{\normalsize 符号}&{\normalsize 指数幂}\\
\midrule
皮[可]  & p             & $10^-12$  &千     &k  &$10^3$\\
纳[诺]  & n             &$10^-9$    &兆     &M  &$10^6$\\     
微     &\textup{$\mu$}  &$10^-6$    &吉[咖] &G  &$10^9$\\
质量    & m             &$10^-3$    &太[拉] &T  &$10^(12)$\\    
\bottomrule
\end{tabular}
\caption[公制词头]{常用公制词头}
\label{tab:prefixes}
\end{table*}

公制词头即度量单位的前缀,用于表示数量级,便于读写和理解。在工程记数法
中,电子和电气工程中使用10种公制词头。表\nameref{tab:prefixes}
列出了最常用的公制词头、符号和相应的10的指数幂。

公制词头仅添加于计量单位符号之前,如伏[特]、安[培]和欧[姆]。例如,$0.025$A 可以用工程记数法表示为
$25\times 10^-3$A,也可以表示为$25$mA ,读作25毫安培。
